\documentclass[11pt]{article}

\usepackage[pages=all, color=black, position={current page.south}, placement=bottom, scale=1, opacity=1, vshift=5mm]{background}

\usepackage[margin=1in]{geometry} % full-width

% AMS Packages
\usepackage{amsmath}
\usepackage{amsthm}
\usepackage{amssymb}
\usepackage{multirow}
% Unicode
\usepackage[utf8]{inputenc}
\usepackage{hyperref}
\hypersetup{
	unicode,
%	colorlinks,
%	breaklinks,
%	urlcolor=cyan, 
%	linkcolor=blue, 
	pdfauthor={Author One, Author Two, Author Three},
	pdftitle={A simple article template},
	pdfsubject={A simple article template},
	pdfkeywords={article, template, simple},
	pdfproducer={LaTeX},
	pdfcreator={pdflatex}
}

% Natbib
\usepackage[sort&compress,numbers,square]{natbib}
% \bibliographystyle{mplainnat}

% Theorem, Lemma, etc
\theoremstyle{plain}
\newtheorem{theorem}{Theorem}
\newtheorem{corollary}[theorem]{Corollary}
\newtheorem{lemma}[theorem]{Lemma}
\newtheorem{claim}{Claim}[theorem]
\newtheorem{axiom}[theorem]{Axiom}
\newtheorem{conjecture}[theorem]{Conjecture}
\newtheorem{fact}[theorem]{Fact}
\newtheorem{hypothesis}[theorem]{Hypothesis}
\newtheorem{assumption}[theorem]{Assumption}
\newtheorem{proposition}[theorem]{Proposition}
\newtheorem{criterion}[theorem]{Criterion}
\theoremstyle{definition}
\newtheorem{definition}[theorem]{Definition}
\newtheorem{example}[theorem]{Example}
\newtheorem{remark}[theorem]{Remark}
\newtheorem{problem}[theorem]{Problem}
\newtheorem{principle}[theorem]{Principle}

\usepackage{graphicx}
\graphicspath{{fig/}}

%\usepackage[linesnumbered,ruled,vlined,commentsnumbered]{algorithm2e} % use algorithm2e for typesetting algorithms
\usepackage{algorithm, algpseudocode} % use algorithm and algorithmicx for typesetting algorithms
\usepackage{mathrsfs} % for \mathscr command

\usepackage{lipsum}



\date{
\today
}

\begin{document}
\title{Secure Non-persistent Instant Messaging}
\author{Dylan Moran, Sydney Pennington, Jordan Rivera, Joey Saline, Mason Wolfe}
\maketitle
	
\begin{abstract}
Please fill out abstract
		
% %		\noindent\textbf{Keywords:} article, template, simple
\end{abstract}

% % 	\tableofcontents

\section{Introduction}\label{sec:intro}
The problem we are trying to solve is building a secure, non-persistent, instant messaging application. The foundation behind this app

\subsection{Importance}\label{sec:importance}
The most valuable asset to have is a user’s data because of the identity it forms. Security is an important factor in instant messaging applications because it prevents access to the User’s information. User information can be logged on a server with their emails, passwords, and location. A compromised server can give access to all this information. Debatably the most important information may come from the chat logs/history where information is sent through various media forms such as images, videos, links, and text that could be personal and sensitive information relating to the user. Having a non-persistent chat implementation ensures the user’s privacy is respected by the administrators and protected from unwanted eyes.

\subsection{Challenges}\label{sec:challenges}
The intended use can pose an ethical issue. A secure encrypted messaging application where chat logs are not stored can pose an ethical issue. Some users can use this to protect themselves from tyrannical governments, blackmail, hackers, or third-party ad targeting. However, this can also be used for malicious purposes where it can serve as a hub for criminals to avoid government or law enforcement from viewing their messages and monitoring their ploys.

\subsection{Attempts}\label{sec:attempts}
Existing solutions to this problem would be What's App, Signal, Grindr, and Zoom. Each of which has some aspects of security.
\subsubsection{Shortcomings}\label{sec:shortcomings}
This is a subsection containing notations.

\section{System Specifications}\label{sec:sysspecs}
This section includes literature review.... As stated in Section \ref{sec:intro}, this is good.

\subsection{User Classes}\label{sec:userclasses}
This is a subsection containing preliminaries.

\subsection{Functional Requirements}\label{sec:funcrequirements}
This is a subsection containing notations.

\subsection{Non-function Requirements}\label{sec:nonfuncrequirements}
This is a subsection containing notations.

\subsection{Operational Requirements}\label{sec:operationalrequirements}
This is a subsection containing notations.

\subsection{Design and Implementation}\label{sec:designandimplementationrequirements}
This is a subsection containing notations.

\end{document}
