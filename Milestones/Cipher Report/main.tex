\documentclass[11pt]{article}
\usepackage[pages=all, color=black, position={current page.south}, placement=bottom, scale=1, opacity=1, vshift=5mm]{background}
\usepackage[margin=1in]{geometry} % full-width
% AMS Packages
\usepackage{amsmath}
\usepackage{amsthm}
\usepackage{amssymb}
\usepackage{multirow}
\usepackage{array}
% Unicode
\usepackage[utf8]{inputenc}
\usepackage{hyperref}
\hypersetup{
	unicode,
%	colorlinks,
%	breaklinks,
%	urlcolor=cyan, 
%	linkcolor=blue, 
	pdfauthor={Author One, Author Two, Author Three},
	pdftitle={A simple article template},
	pdfsubject={A simple article template},
	pdfkeywords={article, template, simple},
	pdfproducer={LaTeX},
	pdfcreator={pdflatex}
}
% Natbib
\usepackage[sort&compress,numbers,square]{natbib}
% \bibliographystyle{mplainnat}
% Theorem, Lemma, etc
\theoremstyle{plain}
\newtheorem{theorem}{Theorem}
\newtheorem{corollary}[theorem]{Corollary}
\newtheorem{lemma}[theorem]{Lemma}
\newtheorem{claim}{Claim}[theorem]
\newtheorem{axiom}[theorem]{Axiom}
\newtheorem{conjecture}[theorem]{Conjecture}
\newtheorem{fact}[theorem]{Fact}
\newtheorem{hypothesis}[theorem]{Hypothesis}
\newtheorem{assumption}[theorem]{Assumption}
\newtheorem{proposition}[theorem]{Proposition}
\newtheorem{criterion}[theorem]{Criterion}
\theoremstyle{definition}
\newtheorem{definition}[theorem]{Definition}
\newtheorem{example}[theorem]{Example}
\newtheorem{remark}[theorem]{Remark}
\newtheorem{problem}[theorem]{Problem}
\newtheorem{principle}[theorem]{Principle}
\usepackage{graphicx}
\graphicspath{{fig/}}
%\usepackage[linesnumbered,ruled,vlined,commentsnumbered]{algorithm2e} % use algorithm2e for typesetting algorithms
\usepackage{algorithm, algpseudocode} % use algorithm and algorithmicx for typesetting algorithms
\usepackage{mathrsfs} % for \mathscr command
\usepackage{lipsum}

\begin{document}
\begin{titlepage}
   \begin{center}
   
    \vspace*{1cm}
    \Huge{
        
        \textbf{Cipher}
    }
    
    \vspace{0.5cm}
    \Huge{
        Secure Non-persistent Instant Messaging
    }
    
    \vspace{1.5cm}
    \large{
        Dylan Moran, Sydney Pennington, Jordan Rivera, Joey Saline, Mason Wolfe
    }

    \vfill
    This project was developed for and under the guidance of Professor Pradeep K. Atrey\ref{} of the University of Albany. Certain aspects of the system were direct requirements specified by Professor Atrey as the sponsor for Cipher.
            
    \vspace{0.8cm}
                 
    Department of Computer Science\\
    University at Albany\\
    United States\\
    \today
            
   \end{center}
\end{titlepage}


% % 	\tableofcontents

\section{Introduction}\label{sec:intro}
As technologies continue to advance at a rapid pace and communications are taking place almost entirely online; what steps can be taken to ensure personal privacy? It is becoming increasingly visible that the personal information you put online is not safe from prying eyes. In the wake of massive hacking scandals, governmental spying, and corporate data collection, we as individuals must evolve and develop technologies that preserves our right to privacy. We aim to develop a secure non persistent messaging applications that incorporates end-to-end encryption. End to End Encryption is used to ensure messages are sent and read securely without third party surveillance. ("What Is End-to-end Encryption?") End to End Encryption can vary in strength depending on the size of the keys, padding, and layers of encryption. Non persistence chats prevent chat history, backlogs, and any trace of messaging to be erased when a message is read.

\subsection{Importance}\label{sec:importance}
The most valuable asset to have is a user’s data because of the identity it forms. Security is an important factor in instant messaging applications because it prevents access to the user’s information. User information is often logged on a server with their emails, passwords, contacts, messages, and IP addresses are stored. A compromised server can give access to all this information. Debatably the most important information may come from the chat logs and contact history where information is sent through various media forms such as images, videos, links, and text that could be personal and sensitive information relating to the user. Having a non-persistent chat implementation ensures the user’s privacy is respected by the administrators and protected from unwanted eyes.

\subsection{Challenges}\label{sec:challenges}
The intended use can pose an ethical issue. A secure encrypted messaging application where chat logs are not stored can pose an ethical issue. Some users can use this to protect themselves from tyrannical governments, blackmail, hackers, or third-party ad targeting. However, this can also be used for malicious purposes where it can serve as a hub for criminals to avoid government or law enforcement from viewing their messages and monitoring their ploys. Also end to end encryption can still be susceptible to man in the middle attacks and key forging so we must develop a hybrid style of encryption, that includes multiple layers of security without effecting performance. (Marlinspike and Perrin)

\subsection{Attempts}\label{sec:attempts}
There are several existing applications that seek to solve this problem. However, they are not without their shortcomings.
\subsubsection{Facebook Messenger}
Facebook Messenger collects and stores almost all user information. Facebook is well known to sell user information and has worked with intelligence agencies in the past. User messages are not encrypted by default, however they have recently added a feature which allows a user to opt in for one to one end to end encryption. (Sarwar)
\subsubsection{Snapchat}
While people perceive Snapchat to be non persistent, they actually store almost all of your data. The messages and pictures may be non persisting your device, however your message history is still saved in Snapchat's cloud. Also pictures are encrypted end to end however, messages are not.
(Dvorak) ("Privacy Policy")
\subsubsection{WhatsApp}
WhatsApp does use end to end encryption, however your metadata is not encrypted which means it is easily accessible by staff and possible hackers who you are talking to, when, and from where. WhatsApp has also worked with intelligence agencies in the past. However, they do not store your chat history on their servers for long. Only if the message cannot be delivered to you right away. Once the message is delivered to your device, it is then deleted. ("About End-to-end Encryption")
\subsubsection{Telegram}
Telegram is not end to end encrypted by default. It is only encrypted from client to server. You can choose to make a secret chat with one other person which will be end to end encrypted. Telegram doesn't support end to end encryption for group chats. They also collect your IP address, device information, and username changes. (Davies) (Pismennaya and Baryshevskiy) (Nesbo)
\newline \newline




\setlength{\tabcolsep}{4pt} %Gap between letters and boxes 
\resizebox{\textwidth}{!}{
\begin{tabular}{|l|c|c|c|c|c|c|} 

 \hline
 \multicolumn{7}{|c|}{Existing Solutions} \\
 \hline
 Features & Facebook Messenger  &  iMessage & Telegram & WhatsApp& Wire & Signal\\
 \hline
 Collects User Data & Yes & Yes & Yes & Yes & No & No \\
 \hline
 Encryption by Default & No & Yes & No & Yes & Yes & Yes \\
 \hline
 Open-source \\(code and server)& No & No & No & No & Yes & Yes\\
 \hline
 Metadata\\ is Encrypted & No & No & No & No & No & Yes\\
 \hline 
 Stores timestamps \\and IP addresses &Yes&Yes & Yes & Yes & Yes &No \\
 \hline 
\end{tabular} 
}

\section{User Classes}\label{sec:userclasses}

\subsection{Room Administrator}
This class of user refers to a user who created a chatroom. Being in this user class would allow a person full control over the chatroom they created. They should be allowed to kick users, invite users, assign roles, set chatroom permissions, and destroy the chatroom. This user class also includes all aspects of the general user class. 
\subsection{Room Member}
This class of user refers to a user who was invited or joined a chatroom. This user class would be allowed the bare minimum amount of control granted by the chatroom administrator. They could send messages, and potentially invite other users if the admin allows it. They are also susceptible to being kicked or blocked from the chatroom. They can however be assigned a role that allows them a higher level of control by the room administrator.  This user class also includes all aspects of the general user class. 
\subsection{General Users}
This class of user refers to all other users that exist outside of chat rooms. General users should be allowed to add friends, block users, see their friend's statuses, create chat rooms and send direct messages to their friends.

\section{Functional Requirements}\label{sec:funcrequirements}
All functional requirements only exist in the scope of the User's own account
\subsection{Encrypt User Messages}
The messages will need to be encrypted from end to end using a popular encryption algorithm (Indeed Editorial Team)
\subsection{Non-Persistance}
A users message history should be deleted from the server after receiving a message. After the user closes the application, the message should also be erased from the client's device.
\subsection{Register User Account}
Users will create an account with a username, password, and email. If the username is taken, or there is already an account registered to the email, the user will be notified and the account not created
\subsection{Change Password}
In the event of a compromised account or a user forgets his/her password, the user should be able to reset their password via their email.
\subsection{Unregister Account}
If a user wants to delete their account, they will be able to do so by following a series of steps available on the app. A user can only delete the account they are logged in on.
\subsection{Set User Presence}
A user will be able to choose whether they want their presence shown to their friends or not. Their presence will be shown by default.
\subsection{Edit User Profile}
A user can edit their profile with an uploaded picture, and a summary about themselves
\subsection{Manage Friend List}
A user should be allowed to add and remove friends
\subsection{View Friends Statuses}
A user should be allowed to see what friends are actively using the app via their statuses; if their friends haven't disabled status viewing. 
\subsection{Manage Blocked Users}
A user should be able to block another user from messaging them, inviting them to chat rooms, and requesting them as a friend. All blocked users will be stored in a blocked users list for easy management and potential unblocking. 
\subsection{Create Chat Room}
A user should be able to create a chatroom. This user will automatically be the administrator of the chatroom, and have the highest level of permissions within the context of the chatroom.
\subsection{Administrate Chatroom}
A user who has the proper permissions (Admin) should be able to kick users, set the chatroom to invite only, closed, or open to join. The Admin of the chatroom should also be allowed to grant users higher level of access, such as allowing them to invite their friends or making them an administrator as well.

\section{Non-functional Requirements}\label{sec:nonfuncrequirements}
\subsection{Performance}
- A user should be able to register an account within 30 seconds given a valid username, 
  password, and email
\newline
- Any message sent to the server whether it be an API request or an XML stanza should reach 
  the server within a second from the push of a button.
\newline
- Given an active client side device, any messages sent to said user should reach them 
  within ten seconds and presented correctly. \newline
- Message should take no longer than 1 minute to reach user \newline
- Messages are correctly sent and received as intended \newline
- Group Chat Message Speed are unaffected by the multiple clients \newline
- Encryption should be at a low expense, AES level performance, RSA level encryption \newline
- Android and IOS compatible, as well as uphold W3 standards

\subsection{Reliability}
- No messages will be lost in transit given an active WiFi connection
  \newline
- The server should always be running, excluding needed maintenance
  \newline
- All web requests submitted will be executed within 5 seconds given the right 
  credentials
    \newline
- Message receival rate greater than 99 \newline 
- Messages are Non-persistent on the Client and Server side \newline
- Application can handle multiple clients at within an allotted time period \newline
- Application can handle message traffic precisely \newline

\subsection{Security}
- All messages sent should be encrypted with the recipients public key and then decrypted 
  via the recipient's private key upon arrival to the recipient's device
    \newline
- There should be limited access to the back end of the messaging application. This 
  includes most ports being closed, excluding the one receiving XML stanza and handling HTTP requests. All requests will be handled via an API and/or a Web Server.
    \newline
- Users must authenticate themselves via a Username and Password to enable requests related 
  to that account
    \newline
- All users only have control over the scope of their account.
  \newline
- Users have a limited amount of requests available to them
  \newline
- Upon retrieval of a user's messages to their client side device, a request will be sent to the server to erase all messages just received. Upon closing of the application, those messages will also be erased from the client device
  \newline
- Hybrid encryption and padding of User Messages \newline
- Encrypted Database protecting all User information \newline
- No stored location Data \newline

  
\subsection{Availability}
- The application should have a publicly available website that almost anyone can access 
  around the globe \newline
- Registration should be quick and straight forward \newline
- The server should always be active excluding maintenance \newline
- Accessible to anybody who has a mobile device with Android and IOS platforms \newline
- Accessible through the Google Play Store or IOS App Store \newline

\subsection{Usability}
- The GUI should be straight forward and not overwhelming or confusing  \newline
- A user should not have to explore the application in depth in order to quickly begin using it  \newline
- Any User with a mobile device that supports Android or IOS can use the application \newline
- Can be used to communicate with multiple clients at once \newline
- Professor Atrey should be able to locate a group chat and join without asking for assistance \newline
- Upon login you can view a new unread message within 10 seconds \newline


\subsection{Scalability}
- The database and server will be easily upgradable with Processing Power, RAM, and SSD in order to support more on coming users.   \newline
- Server should be able to welcome up to 1000 new users per month \newline
- Application should be able to withstand introduction to IOS App Store and Android Google Play \newline
- Application should be able to implement a public domain name on the World Wide Web \newline
- Application should be able to handle larger Chat Rooms and have faster messaging speeds \newline


\section{Operational Requirements}\label{sec:operationalrequirements}
A user needs to have access to a touch screen mobile device, preferably an Android or an IOS device that can install the application. They will also need  internet access on their mobile device to access the application for installation and to connect to the Application's Server. With Internet access, the User can receive and send messages in real time. The mobile device’s permissions should allow internet access and local storage for the application to work. Having touch screen capability helps the User type all their messages with a keyboard and navigate the application with ease.

\section{Design and Implementation}\label{sec:designandimplementationrequirements}
- The android app data needs to be normalized. \newline
- The app will maintain a simple and accessible style. \newline
- The messaging app will have phrasing and language easy to understand for anyone of all 
  ages, groups and fits a variety of situations. \newline
- Users should not have access to chat history after an allotted time or if read \newline
- Application should not have Location Data or personal information \newline
- The application will be restricted to android and IOS platforms for mobile applications \newline
- Users should not have access to User information \newline
- User Information other than Username and Password can not be publicly accessible \newline

\section{References}
(1) "About End-to-end Encryption." Faq.Whatsapp.Com, 1 Jan. 2021, faq.whatsapp.com/820124435853543/. Accessed 3 Feb. 2023. \newline \newline
(2) "What Is End-to-end Encryption?" Www.Ibm.Com, 1 Jan. 2023, www.ibm.com/topics/end-to-end-encryption. Accessed 2 Feb. 2023. \newline \newline
(3) Marlinspike, Moxie, and Perrin, Trevor. "The Double Ratchet Algorithm." Signal.Org, 20 Nov. 2016, signal.org/docs/specifications/doubleratchet/. Accessed 2 Feb. 2023. \newline \newline
(4) "Privacy Policy." Values.Snap.Com, 29 Jun. 2022, values.snap.com/privacy/privacy-policy. Accessed 2 Feb. 2023. \newline \newline
(5) Indeed Editorial Team. "Types of Encryption: 5 Common Encryption Algorithms." Www.Indeed.Com, 21 Jul. 2021, www.indeed.com/career-advice/career-development/types-of-encryption. Accessed 5 Feb. 2023. \newline \newline
(6) Pismennaya, Kate, and Baryshevskiy, Anton. "How to Make a Really Secure Messaging App Similar to Signal?" Themindstudios.Com, 3 Jun. 2022, themindstudios.com/blog/how-to-make-a-secure-chat-app-like-signal/. Accessed 5 Feb. 2023. \newline \newline
(7) Nesbo, Elliot. "Why Telegram Isn't As Secure As You Think It Is." www.makeuseof.Com, 14 Sept. 2021, www.makeuseof.com/telegram-security/v. Accessed 5 Feb. 2023. \newline \newline
(8) Dvorak, Chyelle. "What Data Does Snapchat Collect?" Www.Reviews.Org, 24 Feb. 2022, www.reviews.org/internet-service/what-data-snapchat-collects/. Accessed 5 Feb. 2023. \newline \newline
(9) Sarwar, Nadeem. "How Safe Is Facebook Messenger And How To Make It Even More Secure." Screenrant.Com, 17 Aug. 2021, screenrant.com/facebook-messenger-safety-end-to-encryption-chat-video-messages-explained/. Accessed 4 Feb. 2023. \newline \newline
(10) Davies, Hannah. "Is Telegram Safe? Here’s What Security Experts Have to Say about the App." Www.Trustedreviews.Com, 26 Jan. 2022, www.trustedreviews.com/news/is-telegram-safe-4130553. Accessed 4 Feb. 2023. \newline

\end{document}
